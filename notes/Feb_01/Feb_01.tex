\documentclass{article}


\usepackage[margin=0.6in]{geometry}
\usepackage{amssymb, amsmath, amsfonts}
\usepackage{tabularx}
\usepackage{arydshln}
\usepackage{mathtools}
\usepackage{changepage}
\usepackage{cancel}
\usepackage{physics}
\usepackage{pgf}
\usepackage{enumerate}
\usepackage{placeins}
\usepackage{enumitem}
\usepackage{nth}
\usepackage{array}
\usepackage{tikz}
\usetikzlibrary{arrows,automata}
\usepackage{nicefrac}
\usepackage{pgfplots}
\newcommand{\enth}{$n$th}
\newcommand{\Rl}{\mathbb{R}}
\newcommand{\Cx}{\mathbb{C}}
\newcommand{\sgn}[1]{\text{sgn}\qty[#1]}
\newcommand{\ran}[1]{\text{ran}\qty[#1]}
\newcommand{\E}{\varepsilon}
\newcommand{\qiq}{\qquad \implies \qquad}
\newcommand{\half}{\nicefrac{1}{2}}
\newcommand{\third}{\nicefrac{1}{3}}
\newcommand{\quarter}{\nicefrac{1}{4}}
\newcommand{\f}[3]{#1\ :\ #2 \rightarrow #3}
\newcommand{\Dx}{\Delta x}
\newcommand{\Dt}{\Delta t}
\newcommand{\hot}{\text{h.o.t.}}
\newcommand{\centdiff}{\frac{u_j^{n+1} - u_j^n}{\Dt}}

\newcommand{\tridsym}[3]{
    \qty(\begin{array}{ccccc}
                    #1 & #2 & & & \\
                    #3 & #1 & #2 & & \\
                    & \ddots & \ddots & \ddots &  \\
                    & & #3 & #1 & #2 \\
                    & & & #3 & #1
                \end{array})
}


\DeclareMathOperator*{\esssup}{\text{ess~sup}}

\title{MAT 228B Notes}
\author{Sam Fleischer}
\date{February 1, 2016}

\begin{document}
    \maketitle

    \section{Von Neumann Analysis}

        This is stability analysis of different schemes using Fourier analysis.  The diffusion equation,
        \begin{align*}
            u_t = Du_{xx}
        \end{align*}
        Using Fourier analysis in $x$, we know $\hat{u}_t(\xi,t) = -D\xi^2\hat{u}(\xi,t)$.

        Von Neumann Analysis is used to analyze constant-coefficient linear problems on the whole real line or a periodic domain.

        \subsection{On the Real Line (Infinite 1D Lattice)}

            Let $x_j = j\Dx$ where $j \in \mathbb{Z}$.  We will use the fact that complex exponentials $v_j = e^{i\xi x_j}$ are eigenfunctions of difference operators.  The forward difference operator
            \begin{align*}
                (D_+v)_j \coloneqq \frac{v_{j+1} - v_j}{\Dx}
            \end{align*}
            applied to $v_j = e^{i\xi x_j}$ gives
            \begin{align*}
                (D_+v)_j = \frac{e^{i\xi x_{j+1}} - e^{i\xi x_j}}{\Dx} = e^{i\xi x_j}\underbrace{\qty(\frac{e^{i\xi\Dx} - 1}{\Dx})}_\text{eigenvalue $\lambda_j(\xi)$ for $v_j$}
            \end{align*}
            Note the eigenvalues of $v_j$ are dependent on the wave number $\xi$.
            For the second-difference operator,
            \begin{align*}
                (D^2v)_j = \frac{v_{j-1} + 2v_j + v_{j+1}}{\Dx^2} = \frac{e^{i\xi x_{j-1}} - 2e^{i\xi x_j} + e^{i\xi x_{j+1}}}{\Dx^2} &= e^{i\xi x_j}\underbrace{\qty(\frac{e^{-i\xi\Dx} - 2 + e^{i\xi\Dx}}{\Dx^2})}_\text{eigenvalue} = \underbrace{\frac{2}{\Dx^2}(\cos(\xi\Dx) - 1)}_\text{eigenvalue}v_j
            \end{align*}
            Sometimes this is easy to write as $\lambda = -\frac{4}{\Dx^2}\sin^2(\frac{\xi\Dx}{2})$.

            Let $v_j$ be a discrete function on the discrete lattice $x_j = j\Dx$.  The Fourier transform of the infinite sequence $v_j$ is
            \begin{align*}
                \hat{v}(\xi) = \frac{\Dx}{\sqrt{2\pi}}\sum_{j\in\mathbb{Z}}v_j e^{i\xi x_j}
            \end{align*}
            This looks just like a discretized version of the continuous Fourier Transform.  This is the projection of the function onto the eigenfunctions.  $\xi$ is continuous, but it's bounded.  The highest spacial frequency representable on the discrete mesh with $\Dx$ is $2\Dx$, and the wavelength of $e^{i\xi x_j}$ is (Bob talked really fast), anyway, $2\Dx = \frac{2\pi}{\xi}$, so $\abs{\xi} \leq \frac{\pi}{\Dx}$.

            The inverse Fourier transform is given by
            \begin{align*}
                v_j = \frac{1}{\sqrt{2\pi}}\int_{-\frac{\pi}{\Dx}}^{\frac{\pi}{\Dx}} \hat{v}(\xi)e^{i\xi x_j}\dd\xi
            \end{align*}

            There is a duality between the spaces $\mathbb{Z}$ and $[a,b]$, and between $\Rl$ and $\Rl$.  We will use Parseval's relation, which is $\norm{v(j)}_2 = \norm{\hat{v}(\xi)}_2$.  The ``two-norm'' of the discrete function is
            \begin{align*}
                \norm{v(j)}_2 = \qty(\Dx\sum_{j\in\mathbb{Z}}\abs{v_j}^2)^{\nicefrac{1}{2}}
            \end{align*}
            We can use this for stability analysis:
            \begin{align*}
                u^{n+1} = Bu^n
            \end{align*}
            Previously, we want to control $\norm{B}_2 \leq 1 + \alpha\Dt$ for stability.  Instead, we can show that $\norm{u^{n+1}}_2 \leq \qty(1 + \alpha\Dt)\norm{u^n}_2$.  Why is this equivalent?  Suppose the statement.  Then
            \begin{align*}
                \frac{\norm{u^{n+1}}_2}{\norm{u^n}_2} \leq 1 + \alpha\Dt \implies {\norm{Bu^n}_2}{\norm{u^n}_2} \leq 1 + \alpha\Dt
            \end{align*}
            Since this is true for all $u$, maximize $\norm{B}_2 \leq 1 + \alpha\Dt$.

            So we can look at the norms of the discrete functions (and so the norms of their Fourier transforms) in order to show stability of the scheme.

            If we can show that
            \begin{align*}
                \norm{\hat{u}^{n+1}}_2 \leq \qty(1 + \alpha\Dt)\norm{\hat{u}^n}_2,
            \end{align*}
            then the scheme is stable.

    \section{Forward Euler for Diffusion}

        $$u_j^{n+1} = u_j^n + \frac{\Dt}{\Dx^2}D\qty(u_{j-1}^n - 2u_j^n + u_{j+1}^n).$$
        We can represent $u_j^n$ as
        \begin{align*}
            u_j^n = \frac{1}{\sqrt{2\pi}}\int_{-\nicefrac{\pi}{\Dx}}^{\nicefrac{\pi}{\Dx}}\hat{u}^n(\xi)e^{i\xi x_j}\dd \xi
        \end{align*}
        But, after some manipulation,
        \begin{align*}
            u_j^{n+1} = \frac{1}{\sqrt{2\pi}}\int_{-\nicefrac{\pi}{\Dx}}^{\nicefrac{\pi}{\Dx}}\qty(1 + \frac{\Dt D}{\Dx^2}\qty(e^{-i\xi\Dx} - 2 + e^{i\xi\Dx}))\hat{u}^n(\xi)e^{i\xi x_j}\dd\xi
        \end{align*}
        Now we throw both sides into the Fourier transform:
        \begin{align*}
            \hat{u}^{n+1}(\xi) = \qty(1 + \frac{2\Dt D}{\Dx^2}\qty(\cos(\xi\Dx) - 1))\hat{u}^n(\xi)
        \end{align*}
        This is of the form $\hat{u}^{n+1}(\xi) = g(\xi)\hat{u}^n(\xi)$.  \textbf{Differences have become multiplication}.  $g(\xi)$ is called the amplification factor.  It is the amount by which the Fourier transform gets multiplied by in each step.
        In Fourier space,
        \begin{align*}
            \norm{\hat{u}^{n+1}}_2 = \norm{g(\xi)\hat{u}^n(\xi)}_2 \leq \norm{g(\xi)}_\infty\norm{\hat{u}^n(\xi)}_2 && \text{using H$\ddot{\text{o}}$lder's Inequality}
        \end{align*}
        If we can show that $\displaystyle\max_\xi\abs{g(\xi)} \leq 1 + \alpha \Dt$, then the scheme is stable.  So,
        \begin{align*}
            g(\xi) = 1 - \frac{4D\Dt}{\Dx^2}\sin^2\qty(\frac{\xi\Dx}{2})
        \end{align*}
        So
        \begin{align*}
            \abs{g(\xi)} \leq 1 \qquad \text{for all } \xi
        \end{align*}
        means
        \begin{align*}
            -1 \leq 1 - \frac{4D\Dt}{\Dx^2}\sin^2\qty(\frac{\xi\Dx}{2}) \leq 1.
        \end{align*}
        This requires that
        \begin{align*}
            0 \leq \frac{4D\Dt}{\Dx^2}\sin^2\qty(\frac{\xi\Dx}{2}) \leq 2
        \end{align*}
        This holds for all $\xi$ if
        \begin{align*}
            \Dt \leq \frac{\Dx^2}{2D}
        \end{align*}

\end{document}



















