\documentclass{article}


\usepackage[margin=0.6in]{geometry}
\usepackage{amssymb, amsmath, amsfonts}
\usepackage{tabularx}
\usepackage{arydshln}
\usepackage{mathtools}
\usepackage{changepage}
\usepackage{asymptote}
\usepackage{cancel}
\usepackage{physics}
\usepackage{pgf}
\usepackage{enumerate}
\usepackage{placeins}
\usepackage{enumitem}
\usepackage{nth}
\usepackage{array}
\usepackage{tikz}
\usetikzlibrary{arrows,automata}
\tikzset{
  saveuse path/.code 2 args={
    \pgfkeysalso{#1/.style={insert path={#2}}}%
    \global\expandafter\let\csname pgfk@\pgfkeyscurrentpath/.@cmd\expandafter\endcsname
      % not optimal as it is now global through out the document
                           \csname pgfk@\pgfkeyscurrentpath/.@cmd\endcsname
    \pgfkeysalso{#1}},
  /pgf/math set seed/.code=\pgfmathsetseed{#1}}
\usepackage{nicefrac}
\usepackage{pgfplots}
\newcommand{\enth}{$n$th}
\newcommand{\Rl}{\mathbb{R}}
\newcommand{\Cx}{\mathbb{C}}
\newcommand{\sgn}[1]{\text{sgn}\qty[#1]}
\newcommand{\ran}[1]{\text{ran}\qty[#1]}
\newcommand{\E}{\varepsilon}
\newcommand{\qiq}{\qquad \implies \qquad}
\newcommand{\half}{\nicefrac{1}{2}}
\newcommand{\third}{\nicefrac{1}{3}}
\newcommand{\quarter}{\nicefrac{1}{4}}
\newcommand{\f}[3]{#1\ :\ #2 \rightarrow #3}
\newcommand{\Dx}{\Delta x}
\newcommand{\Dy}{\Delta y}
\newcommand{\Dt}{\Delta t}
\newcommand{\hot}{\text{h.o.t.}}
\newcommand{\centdiff}{\frac{u_j^{n+1} - u_j^n}{\Dt}}

\newcommand{\tridsym}[3]{
    \qty(\begin{array}{ccccc}
                    #1 & #2 & & & \\
                    #3 & #1 & #2 & & \\
                    & \ddots & \ddots & \ddots &  \\
                    & & #3 & #1 & #2 \\
                    & & & #3 & #1
                \end{array})
}


\DeclareMathOperator*{\esssup}{\text{ess~sup}}

\title{MAT 228B Notes}
\author{Sam Fleischer}
\date{February 10, 2017}

\begin{document}
    \maketitle

    \section{Last Time}
        \begin{itemize}
            \item Laplacian in 2D: $\laplacian u = u_{xx} + u_{yy}\qquad L = L_x + L_y$.
            \item LOD method.. think about as a fractional stepping approach
            \item Method of lines: $\frac{\dd u}{\dd t} = L_x u + L_y u$.  Solve $\frac{\dd u}{\dd t} = L_xu$ and then $\frac{\dd u}{\dd t} = L_yu$.
        \end{itemize}

    \section{ADI (class of) Schemes}
        \begin{itemize}
            \item This is another approach ``Alternating Direction Implicit''
            \item For $2D$ diffusion, a common (Peaceman-Rachford) ADI scheme is
            \begin{align*}
                \qty(I - \frac{d\Dt}{2}L_x)u^* = \qty(I + \frac{d\Dt}{2}L_y)u^n
            \end{align*}
            \item Kind of looks like Crank Nicolson.. but $y$ explicit and $x$ implicit at the same time (for the intermediate step)
            \item The follow this up with the secondary step:
            \begin{align*}
                \qty(I - \frac{d\Dt}{2}L_y)u^{n+1} = \qty(I + \frac{d\Dt}{2}L_x)u^*.
            \end{align*}
            \item What is the work?
            \begin{itemize}
                \item Step 1: solve $N_y$ tridiagonal systems of size $N_x$ (work is $\order{N_xN_y} = \order{N}$, which is optimal, ($N$ is the number of gridpoints in $2D$))
                \item Step 2:solve $N_x$ tridiagonal systems of size $N_y$ (work is $\order{N_xN_y} = \order{N}$)
                \item So the total work is $\order{N}$.
            \end{itemize}
            \item Is it stable?
            \begin{itemize}
                \item Doing the discrete Fourier transform gives
                \begin{align*}
                    \qty(1 + 2b\Dt\sin^2(\frac{\xi_1\Dx}{2}))\hat{u^*} &= \qty(1 - 2b\Dt\sin^2(\frac{\xi_2\Dy}{2}))\hat{u^n} \\
                    \qty(1 + 2b\Dt\sin^2(\frac{\xi_1\Dy}{2}))\hat{u^{n+1}} &= \qty(1 - 2b\Dt\sin^2(\frac{\xi_2\Dx}{2}))\hat{u^*}
                \end{align*}
                \item Solving this gives
                \begin{align}
                    \hat{u^{n+1}} = \underbrace{\frac{\qty(1 - 2b\Dt\sin^2(\frac{\xi_1\Dx}{2}))\qty(1 - 2b\Dt\sin^2(\frac{\xi_2\Dy}{2}))}{\qty(1 + 2b\Dt\sin^2(\frac{\xi_2\Dy}{2}))\qty(1 + 2b\Dt\sin^2(\frac{\xi_2\Dx}{2}))}}_{g(\xi_1,\xi_2)}\hat{u^n}
                \end{align}
                \item Note that $g(\xi_1,\xi_2) = g_1(\xi_1)g_2(\xi_2)$ where $g_1$ and $g_2$ are the ampification factors of Crank-Nicolson.
                \item So, $\abs{g(\xi_1,\xi_2)} = \abs{g_1(\xi_1)}\cdot\abs{g_2(\xi_2)} \leq 1$, so it is unconditionally stable.
            \end{itemize}
            \item Is it consistent?
            \begin{itemize}
                \item Multiply first equation by $\qty(I + \frac{b\Dt}{2}L_x)$, so.
                \begin{align}
                    \underbrace{\qty(I + \frac{b\Dt}{2}L_x)\qty(I - \frac{b\Dt}{2}L_x)}_{\text{these commute}}u^* = \qty(I + \frac{b\Dt}{2}L_x)\qty(I + \frac{b\Dt}{2}L_y)u^n
                \end{align}
                \item So,
                \begin{align}
                    \qty(I - \frac{b\Dt}{2}L_x)\qty(I + \frac{b\Dt}{2}L_x)u^* = \qty(I + \frac{b\Dt}{2}L_x)\qty(I + \frac{b\Dt}{2}L_y)u^n
                \end{align}
                \item Then we eliminate $u^*$ by using the second step of the scheme.
                \begin{align}
                    \qty(I - \frac{b\Dt}{2}L_x)\qty(I - \frac{b\Dt}{2}L_y)u^{n+1} = \qty(I + \frac{b\Dt}{2}L_x)\qty(I + \frac{b\Dt}{2}L_y)u^n
                \end{align}
                \item Expanding this gives
                \begin{align}
                    \qty(I - \frac{b\Dt}{2}L_x - \frac{b\Dt}{2}L_y + \frac{b^2\Dt^2}{4}L_xL_y)u^{n+1} = \qty(I + \frac{b\Dt}{2}L_x + \frac{b\Dt}{2}L_y + \frac{b^2\Dt^2}{4}L_xL_y)u^n
                \end{align}
                \item So this is a small perturbation of Crank-Nicolson.
                \begin{align}
                    \underbrace{\qty(I - \frac{b\Dt}{2}L)u^{n+1} = \qty(I + \frac{b\Dt}{2}L)u^n}_\text{Crank-Nicolson, where $L$ is the discrete 2D Laplacian} - \frac{b^2\Dt^2}{4}L_xL_y\qty(u^{n+1} - u^n)
                \end{align}
                \item So,
                \begin{align}
                    \frac{u^{n+1} - u^n}{\Dt} = \frac{b}{2}Lu^n + \frac{b}{2}Lu^{n+1} - \frac{b^2\Dt^2}{4}L_xL_y\qty(\frac{u^{n+1} - u^n}{\Dt})
                \end{align}
                \item How big is this (do a Taylor Series)?
                \item We know plugging in the solution to the PDE into
                \begin{align}
                    \frac{u^{n+1} - u^n}{\Dt} = \frac{b}{2}Lu^n + \frac{b}{2}Lu^{n+1}
                \end{align}
                gives us LTE for CN (which is second-order).
                \item In the limit of $\Dt,\Dx \rightarrow 0$,
                \begin{align}
                    \Dt^2L_xL_y\qty(\frac{u^{n+1} - u^n}{\Dt}) \rightarrow \Dt^2\frac{\partial^5}{\partial y^2\partial x^2 \partial t}u
                \end{align}
                \item These extra terms contribute $\order{\Dt^2}$ to the truntation error in addition to the LTE of CN.
            \end{itemize}
            \item Thus, the Lax Equivalence Theorem says this is convergent.  It is second-order in space and time and unonditionally stabl $\order{N}$ work per timestep.
            \item What are the boundary conditions shoud hold for $u^*$?
            \begin{itemize}
                \item The first step is Forward Euler in $y$ direction for half a timestep, and
                \item Backward Euler in the $x$ direction for half a timestep.
                \item It's a consistent discretization of the PDE to take step length $\frac{\Dt}{2}$.
                \item So, $u^* = u^{n+\half} + \order{\Dt^2}$.
                \item This means the single-step error in Forward or Backward Euler is $\order{\Dt^2}$.
            \end{itemize}
        \end{itemize}

\end{document}












