\documentclass{article}


\usepackage[margin=0.6in]{geometry}
\usepackage{amssymb, amsmath, amsfonts}
\usepackage{tabularx}
\usepackage{arydshln}
\usepackage{mathtools}
\usepackage{cancel}
\usepackage{physics}
\usepackage{pgf}
\usepackage{enumerate}
\usepackage{placeins}
\usepackage{enumitem}
\usepackage{nth}
\usepackage{array}
\usepackage{tikz}
\usetikzlibrary{arrows,automata}
\usepackage{nicefrac}
\usepackage{pgfplots}
\newcommand{\enth}{$n$th}
\newcommand{\Rl}{\mathbb{R}}
\newcommand{\Cx}{\mathbb{C}}
\newcommand{\sgn}[1]{\text{sgn}\qty[#1]}
\newcommand{\ran}[1]{\text{ran}\qty[#1]}
\newcommand{\E}{\varepsilon}
\newcommand{\qiq}{\qquad \implies \qquad}
\newcommand{\half}{\nicefrac{1}{2}}
\newcommand{\third}{\nicefrac{1}{3}}
\newcommand{\quarter}{\nicefrac{1}{4}}
\newcommand{\f}[3]{#1\ :\ #2 \rightarrow #3}

\newcommand{\tridsym}[3]{
    \qty(\begin{array}{ccccc}
                    #1 & #2 & & & \\
                    #3 & #1 & #2 & & \\
                    & \ddots & \ddots & \ddots &  \\
                    & & #3 & #1 & #2 \\
                    & & & #3 & #1
                \end{array})
}


\DeclareMathOperator*{\esssup}{\text{ess~sup}}

\title{MAT 228B Notes}
\author{Sam Fleischer}
\date{January 9, 2016}

\begin{document}
    \maketitle

    \section{Review}

        \begin{itemize}
            \item 228A
            \begin{itemize}
                \item $\laplacian u = f$
                \item how to discretize (mostly finite difference methods)
                \item Then, how to solve the discretized algebraic equations
                \item There was some theory about accuracy and convergence
            \end{itemize}
            \item 228B
            \begin{itemize}
                \item time-dependent problems
                \item $u_t = Du_{xx}$ (diffusion or heat equation)
                \item $u_t + au_x = 0$ (advection equation)
                \item $u_{tt} = c^2u_{xx}$ (wave equation, really equivalent to solving two advection equations)
                \item Mixed Equations (e.g.~advection-diffusion, etc.)
                \begin{itemize}
                    \item $u_t + au_x = Du_{xx} + R(u)$ (advection-diffusion-reaction equations)
                    \item schemes used to attack problems depend on parameters.. does it look more like one or the other
                \end{itemize}
                \item $u_t + uu_x = Du_{xx}$ (Burger's Equation, i.e.~nonlinear advection).  With $D = 0$ (inviscid), the solution can develop jumps in finite time from smooth initial data.
                \item $\rho\qty(u_t + u\cdot\grad u) = -\grad p + \mu\laplacian u$\ \ with \ \ $\div u = 0$ (Incompressible Navier-Stokes)
            \end{itemize}
        \end{itemize}

    \section{Conservation Laws}

        \begin{itemize}
            \item $u_t + \qty(f(u))_x = 0$ (1D conservation law)
            \item $u_t + \div \qty(F(u)) = 0$ (higher dimension conservation law)
        \end{itemize}

        Let $\rho(x,t)$ be a density, e.g.~$\frac{\text{mass}}{\text{length}}$ where length is 1D volume.  Let $f(\rho)$ be a flux function, i.e.~the rate of stuff moving through surface (in 1D, ``surface'' is point).  In 1D, we have $\frac{\text{mass}}{\text{time}}$ moving through a point.  In 3D, we have $\frac{\text{mass}}{\text{time}\cdot\text{area}}$.

        Consider an interval $[x_1,x_2]$.  Let $A$ be the amount of stuff in $[x_1,x_2]$.  Then $$A(t) = \int_{x_1}^{x_2}\rho(x,t)\dd x.$$
        So we can derive an ODE for $A$, by $$\frac{\dd A}{\dd t} = \frac{\dd}{\dd t}\int_{x_1}^{x_2}\rho(x,t)\dd x.$$  By sign convention, positive is to the right, so $$\frac{\dd A}{\dd t} = \frac{\dd}{\dd t}\int_{x_1}^{x_2}\rho(x,t)\dd x = -f(\rho(x_2,t)) + f(\rho(x_1,t)).$$  Then the fundamental theorem of calculus gives $$\frac{\dd A}{\dd t} = -\int_{x_1}^{x_2}\qty(f(\rho))_x\dd x.$$  We are assuming $\rho$ is nice enough to bring $\frac{\dd}{\dd t}$ through the integral, so
        \begin{align*}
            \int_{x_1}^{x_2}\qty(\rho_t + \qty(f(\rho))_x)\dd x = 0,
        \end{align*}
        which is the integral form of a conservation law.  Since $x_1$ and $x_2$ is arbitrary, we can argue, provided $\rho$ is nice enough (continuously differentiable will do it, for example), then we can drop the integral to get
        \begin{align*}
            \rho_t + \qty(f(\rho))_x = 0
        \end{align*}
        is the differential form of a conservation law.

        Let $u$ be a chemical concentration.  Suppose $u$ is transported by a velocity $a$.  Then the flux function is, in this case, is $f(u) = au$ where $[a] = \frac{\text{length}}{\text{time}}$ and $[u] = \frac{\text{amount}}{\text{length}}$.  Then $u_t + (au)_x = 0$, which is the advection equation (if $a$ is constant, we can get $u_t + au_x = 0$).

        If there is no background flow (advection) but there is diffusion, then $f(u) = -Du_{x}$, which says that things move down the gradient.  This is diffusive flux.  Putting this in the conservation law gives
        \begin{align*}
            u_t + (-Du_x)_x = 0
        \end{align*}
        So if $D$ is constant, we get $u_t = Du_{xx}$.

        \subsection{An example of a system of nonlinear conservation laws}
            Euler equations
            \begin{itemize}
                \item conservation of mass $\rho_t + (v\rho)_x = 0$
                \item conservation of momentum $(\rho v)_t + \qty(\rho v^2 + \rho)_x = 0$
                \item conservation of energy $E_t + \qty(v(E + \rho))_x = 0$
            \end{itemize}

\end{document}



















