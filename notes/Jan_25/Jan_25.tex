\documentclass{article}


\usepackage[margin=0.6in]{geometry}
\usepackage{amssymb, amsmath, amsfonts}
\usepackage{tabularx}
\usepackage{arydshln}
\usepackage{mathtools}
\usepackage{cancel}
\usepackage{physics}
\usepackage{pgf}
\usepackage{enumerate}
\usepackage{placeins}
\usepackage{enumitem}
\usepackage{nth}
\usepackage{array}
\usepackage{tikz}
\usetikzlibrary{arrows,automata}
\usepackage{nicefrac}
\usepackage{pgfplots}
\newcommand{\enth}{$n$th}
\newcommand{\Rl}{\mathbb{R}}
\newcommand{\Cx}{\mathbb{C}}
\newcommand{\sgn}[1]{\text{sgn}\qty[#1]}
\newcommand{\ran}[1]{\text{ran}\qty[#1]}
\newcommand{\E}{\varepsilon}
\newcommand{\qiq}{\qquad \implies \qquad}
\newcommand{\half}{\nicefrac{1}{2}}
\newcommand{\third}{\nicefrac{1}{3}}
\newcommand{\quarter}{\nicefrac{1}{4}}
\newcommand{\f}[3]{#1\ :\ #2 \rightarrow #3}
\newcommand{\Dx}{\Delta x}
\newcommand{\Dt}{\Delta t}
\newcommand{\hot}{\text{h.o.t.}}
\newcommand{\centdiff}{\frac{u_j^{n+1} - u_j^n}{\Dt}}

\newcommand{\tridsym}[3]{
    \qty(\begin{array}{ccccc}
                    #1 & #2 & & & \\
                    #3 & #1 & #2 & & \\
                    & \ddots & \ddots & \ddots &  \\
                    & & #3 & #1 & #2 \\
                    & & & #3 & #1
                \end{array})
}


\DeclareMathOperator*{\esssup}{\text{ess~sup}}

\title{MAT 228B Notes}
\author{Sam Fleischer}
\date{January 25, 2016}

\begin{document}
    \maketitle

    \section{PDE Theory - Consistency, Convergence, and Stability}

        \subsection{Convergence}

            \begin{itemize}
                \item In a nutshell, a convergent scheme gives more accurate answers for finer meshes.
                \item Definition: A numerical method is ``convergent'' if $\forall (x^*, t^*)$ in the domain, $\norm{u_j^n - u(x_j,t_n)} \rightarrow 0$ as $\Dx,\Dt \rightarrow 0$ and $x_j \rightarrow x^*$ and $t_n \rightarrow t^*$.
                \item Note: sometimes we must constrain $\Dx$ and $\Dt$ as they go to $0$.  An example is the Forward Euler for diffusion, where we require $\Dt \leq \dfrac{(\Dx)^2}{2D}$ so the solution doesn't blow up.
                \item The main point is that the timestep and spacestep are related by a constraint that ensures the solution doesn't dlow up.
                \item The definition requires specifying a norm, like the discrete $1$-, $2$-, or $\infty$-norms.  We'll see examples which converge in one norm but not others.
                \item Convergence looks at the error in the solution of a particular problem, while...
            \end{itemize}

        \subsection{Consistency}

            \begin{itemize}
                \item Local Truncation Error (LTE) is the discretization error, that is, the error of the discretization scheme itself.  It is how well differences approximate derivatives.
                \item Consistency looks at the error in the numerical scheme, not an individual solution to a problem.
                \item Definition, a scheme is ``consistent'' if the LTE $\rightarrow 0$ as $\Dx,\Dt \rightarrow 0$.
                \item Example: Forward Euler for Diffusion:
                \begin{align*}
                    \centdiff = \frac{D}{(\Dx)^2}\qty(u_{j-1}^n - 2u_j^n + u_{j+1}^n)
                \end{align*}
                Let $u(x,t)$ solve $u_t = Du_{xx}$.  Let $\tau_j^n$ be the LTE at $(x_j,t_n)$, that is,
                \begin{align}
                    \tau_j^n = \frac{u(x_j,t_n + \Dt) - u(x_j,t_n)}{\Dt} - \frac{D}{(\Dx)^2}\qty(u(x_j-\Dx,t_n) - 2u(x_j,t_n) + u(x_j+\Dx,t_n)).
                \end{align}
                We are interested in how this acts as $\Dx,\Dt \rightarrow 0$, so we look at the Taylor expansions and see that
                \begin{align}
                    \tau_j^n = \frac{\Dt}{2}u_{tt}(x_j,t_n) - \frac{D}{12}\qty(\Dx)^2u_{xxxx}(x_j,t_n) + \hot
                \end{align}
                The take-away:
                \begin{align}
                    \tau = \order{\Dt} + \order{(\Dx)^2}.
                \end{align}
            \end{itemize}

        \subsection{Stability}

            \begin{itemize}
                \item How do we relate LTE (consistency) and convergence?  Through ``stability.''
                \item The Lax-Equivalence Theorem (a.k.a.~the Fundamental Theorem of Finite Differences) says:
                \begin{itemize}
                    \item A linear, consistent difference scheme to a well-posed linear PDE is convergent if and only if it is stable.
                    \item In practice, we will use
                    \begin{align}
                        \text{stability} + \text{consistency} \implies \text{convergence} \qquad \text{(for linear PDEs and linear schemes)}
                    \end{align}
                \end{itemize}
                \item Definition: Consider the linear update rule $u^{n+1} = Bu^n + b^n$.  Let $u^n$ and $v^n$ be two different solutions to the difference scheme (that is, derived from $u^0$ not necessarily equal to $v^0$).  The method is ``stable'' if $\forall T > 0$, $\exists$ a constant $K_T$ such that $\norm{u^n - v^n} \leq K_T\norm{u^0 - v^0}$, independent of $u^0$, $v^0$, for all $n\Dt \leq T$.
                \begin{itemize}
                    \item Note: This is independent of the number of timesteps needed to reach $T$, which is an actual time.
                    \item Another note: The scheme could be more general, i.e.~$B_1u^{n+1} = B_2u^n + \Dt f^n$, but if either $B_1$ or $B_2$ is invertible, it is equivalent.
                \end{itemize}
                \item Alternate Definition (Lax-Richtmyer): The scheme is ``stable'' if  $\forall T > 0$, $\exists$ a constant $C_T > 0$ independent of $\Dt$ such that $\norm{B^n} \leq C_T$ dor all $n\Dt \leq T$ (remember $B$ depends on $\Dt$ and $\Dx$).
            \end{itemize}


\end{document}



















