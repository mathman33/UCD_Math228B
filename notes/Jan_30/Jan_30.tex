\documentclass{article}


\usepackage[margin=0.6in]{geometry}
\usepackage{amssymb, amsmath, amsfonts}
\usepackage{tabularx}
\usepackage{arydshln}
\usepackage{mathtools}
\usepackage{changepage}
\usepackage{cancel}
\usepackage{physics}
\usepackage{pgf}
\usepackage{enumerate}
\usepackage{placeins}
\usepackage{enumitem}
\usepackage{nth}
\usepackage{array}
\usepackage{tikz}
\usetikzlibrary{arrows,automata}
\usepackage{nicefrac}
\usepackage{pgfplots}
\newcommand{\enth}{$n$th}
\newcommand{\Rl}{\mathbb{R}}
\newcommand{\Cx}{\mathbb{C}}
\newcommand{\sgn}[1]{\text{sgn}\qty[#1]}
\newcommand{\ran}[1]{\text{ran}\qty[#1]}
\newcommand{\E}{\varepsilon}
\newcommand{\qiq}{\qquad \implies \qquad}
\newcommand{\half}{\nicefrac{1}{2}}
\newcommand{\third}{\nicefrac{1}{3}}
\newcommand{\quarter}{\nicefrac{1}{4}}
\newcommand{\f}[3]{#1\ :\ #2 \rightarrow #3}
\newcommand{\Dx}{\Delta x}
\newcommand{\Dt}{\Delta t}
\newcommand{\hot}{\text{h.o.t.}}
\newcommand{\centdiff}{\frac{u_j^{n+1} - u_j^n}{\Dt}}

\newcommand{\tridsym}[3]{
    \qty(\begin{array}{ccccc}
                    #1 & #2 & & & \\
                    #3 & #1 & #2 & & \\
                    & \ddots & \ddots & \ddots &  \\
                    & & #3 & #1 & #2 \\
                    & & & #3 & #1
                \end{array})
}


\DeclareMathOperator*{\esssup}{\text{ess~sup}}

\title{MAT 228B Notes}
\author{Sam Fleischer}
\date{January 30, 2016}

\begin{document}
    \maketitle

    \section{Last Time}

        We were analyzing stability, and wrote $u^{n+1} = Bu^n + b^n$.  We know this is stable if $\norm{B^n} \leq C_T$ where $C_T$ is independent of $\Dt$.  Last time we showed (for certain examples) $\norm{B} \leq 1$.  Then $\norm{B^n} \leq 1$.

    \section{What if the Solution is Supposed to Grow in Time?}

        It would be silly to get a numerical scheme which doesn't allow growth.. we need to cautiously allow some growth.  If there is a constant $\alpha \geq 0$ independent of $\Dt$ (for $\Dt$ small enough) such that $\norm{B} \leq 1 + \alpha\Dt$, then the scheme is Lax-Richtmyer Stable.
        \begin{adjustwidth}{1in}{1in}
            Show this is true?  Suppose $\norm{B} \leq 1 + \alpha \Dt$.  We want to show $\norm{B}^n \leq C_T$.  $\norm{B^n} \leq \norm{B}^n \leq (1 + \alpha\Dt)^n \leq e^{\alpha\Dt n} = e^{\alpha T}$.  $\alpha$ is like the exponential growth rate we are going to allow for the problem.
        \end{adjustwidth}
        Consider $u_t = u_{xx} + Ku$.  Spatial diffusion and exponential growth (if $K > 0$).  Consider Forward Euler Stability in $\norm{\cdot}_\infty$.  Forward Euler is
        \begin{align*}
            u^{n+1} = \underbrace{\qty(I + \Dt L + K\Dt I)}_Bu^n
        \end{align*}
        We get $\norm{B}_\infty = \abs{\frac{\Dt}{\Dx^2}} + \norm{1 - \frac{2\Dt}{\Dx^2} + k\Dx} + \abs{\frac{\Dt}{\Dx^2}} \leq 2\frac{\Dt}{\Dx^2} \abs{1 - \frac{2\Dt}{\Dx^@}} + \abs{K}\Dt$.  So, we require $1 - \frac{2\Dt}{\Dx^2} \geq 0$, whege $\frac{\Dt\Dx^2}{2}$.  So, $\norm{B}_\infty \leq 1 + \abs{k}\Dt.$  With the restriction, this method is stable.

        \subsection{If $K < 0$, We Expect the Physical Solution to Decay}

            We want $\norm{B}_\infty \leq 1$.  Try $\Dt$ such that $1 - \frac{2\Dt}{\Dx^2} + k\Dt \geq 0$, i.e.
            \begin{align*}
                \Dt \leq \frac{\Dx^2}{2 - K\Dx^2}
            \end{align*}
            So,
            \begin{align*}
                \norm{B}_\infty = 1 + K\Dt
            \end{align*}
            Want $0 \leq 1 + K\Dt \leq 1$, i.e.~$-1 \leq K\Dt \leq 0$, i.e.~$\Dt \leq -\frac{1}{K}$.

    \section{Variable Coefficient Diffusion}

        In the conservative form, $u_t = \qty(a(x)u_x)_x$.  In multi-D,
        \begin{align*}
            u_t = -\div{J} \qquad \text{where} \qquad J = -a(x)\grad u
        \end{align*}
        We'll generally want to discretize the conservative form.

        \subsection{Discretize the Conservative Form}

            We discretize space (1D, equally spaced).  We have $x_j$ (points) and $x_{j-\half}$ edges.  To approximate the flux $J$,
            \begin{align*}
                 J = -a(x)u_x \qquad \text{where} \qquad J_{j-\half} = - \qty(a(x_{j-\half})\qty(\frac{u_{j} - u_{j-1}}{\Dx}))
            \end{align*}
            So, for $u_t = -J_x$, (in semi-discrete form) we have
            \begin{align*}
                \frac{\dd}{\dd t}u(x_j) = -(\frac{J_{j+\half} - J_{j-\half}}{\Dx})
            \end{align*}
            So,
            \begin{align*}
                \qty[\qty(a(x)u_x)_x]_j = -\qty(\frac{J_{j+\half} - J_{j-\half}}{\Dx}) = \frac{a_{j+\half}\qty(\dfrac{u_{j+1} - u_j}{\Dx}) - a_{j-\half}\qty(\dfrac{u_j - u_{j-1}}{\Dx})}{\Dx} \\
                = \frac{a_{j-\half}u_{j-1} - \qty(a_{j-\half} + a_{j+\half})u_j + a_{j+\half}u_{j+1}}{\Dx^2}
            \end{align*}
            and note that if $a$ is constant it reduces to what we had before.

        \subsection{Stability of Forward Euler in $\norm{\cdot}_\infty$}

            Using a ``constant coefficient way of thinking,'' Bob would guess $\Dt \leq \dfrac{\Dx^2}{2\max(a(x))}$..

            Forward Euler for this problem is
            \begin{align*}
                u^{n+1} = \underbrace{\qty(I + \Dt A)}_Bu^n
            \end{align*}
            \begin{align*}
                \norm{B}_\infty = \max_j\qty(\abs{\frac{a_{j-\half}\Dt}{\Dx^2}} + \abs{1 - \frac{\Dt\qty(a_{j-\half} + a_{j+\half})}{\Dx^2}} + \abs{\frac{a_{j+\half}\Dt}{\Dx^2}})
            \end{align*}
            So we pick $\Dt$ so that $1 - \frac{\Dt\qty(a_{j-\half} + a_{j+\half})}{\Dx^2} \geq 0$ for all $j$, which is equivalent to
            \begin{align*}
                \Dt \leq \frac{\Dx^2}{a_{j-\half} + a_{j+\half}} \qquad \text{for all $j$}
            \end{align*}
            and stuff cancels and we get $\norm{B}_\infty \leq 1$.


\end{document}



















