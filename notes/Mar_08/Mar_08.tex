\documentclass{article}


\usepackage[margin=0.6in]{geometry}
\usepackage{amssymb, amsmath, amsfonts}
\usepackage{tabularx}
\usepackage{arydshln}
\usepackage{mathtools}
\usepackage{changepage}
\usepackage{asymptote}
\usepackage{cancel}
\usepackage{physics}
\usepackage{pgf}
\usepackage{enumerate}
\usepackage{placeins}
\usepackage{nth}
\usepackage{array}
\usepackage{tikz}
\usetikzlibrary{arrows,automata}
\tikzset{
  saveuse path/.code 2 args={
    \pgfkeysalso{#1/.style={insert path={#2}}}%
    \global\expandafter\let\csname pgfk@\pgfkeyscurrentpath/.@cmd\expandafter\endcsname
      % not optimal as it is now global through out the document
                           \csname pgfk@\pgfkeyscurrentpath/.@cmd\endcsname
    \pgfkeysalso{#1}},
  /pgf/math set seed/.code=\pgfmathsetseed{#1}}
\usepackage{nicefrac}
\usepackage{pgfplots}
\usepgfplotslibrary{polar}
\pgfplotsset{holdot/.style={fill=white,only marks,mark=*}}
\pgfplotsset{soldot/.style={only marks,mark=*}}
\newcommand{\enth}{$n$th}
\newcommand{\Rl}{\mathbb{R}}
\newcommand{\Cx}{\mathbb{C}}
\newcommand{\sgn}[1]{\text{sgn}\qty[#1]}
\newcommand{\ran}[1]{\text{ran}\qty[#1]}
\newcommand{\E}{\varepsilon}
\newcommand{\qiq}{\qquad \implies \qquad}
\newcommand{\half}{\nicefrac{1}{2}}
\newcommand{\third}{\nicefrac{1}{3}}
\newcommand{\quarter}{\nicefrac{1}{4}}
\newcommand{\f}[3]{#1\ :\ #2 \rightarrow #3}
\newcommand{\Dx}{\Delta x}
\newcommand{\Dy}{\Delta y}
\newcommand{\Dt}{\Delta t}
\newcommand{\Dphi}{\Delta \phi}
\newcommand{\hot}{\text{h.o.t.}}
\newcommand{\centdiff}{\frac{u_j^{n+1} - u_j^n}{\Dt}}
\newcommand{\dod}{Domain of Depdendence}

\newcommand{\tridsym}[3]{
    \qty(\begin{array}{ccccc}
                    #1 & #2 & & & \\
                    #3 & #1 & #2 & & \\
                    & \ddots & \ddots & \ddots &  \\
                    & & #3 & #1 & #2 \\
                    & & & #3 & #1
                \end{array})
}


\DeclareMathOperator*{\esssup}{\text{ess~sup}}

\title{MAT 228B Notes}
\author{Sam Fleischer}
\date{March 8, 2017}

\begin{document}
    \maketitle

    \section{Systems of Equations}

        Consider $u_t + Au_x = 0$ where $A$ is a diagonalizable matrix with real eigenvalues.  Then $A = V\Lambda V^{-1}$, so
        \begin{align*}
            u_t + V\Lambda V^{-1}u_x &= 0 \\
            V^{-1}u_t + \Lambda V^{-1}u_x &= 0
        \end{align*}
        Defining $w = V^{-1}u$ gives
        \begin{align*}
            w_t + \Lambda w_x = 0
        \end{align*}
        which is a set of de-coupled advection equations (one for each eigenspace).  So upwind, use the sign of the elements of $\lambda$.  Now define $\Lambda^+$ and $\Lambda^-$ as
        \begin{align*}
            \Lambda^+ = \frac{\Lambda + \abs{\Lambda}}{2} \\
            \Lambda^- = \frac{\Lambda - \abs{\Lambda}}{2}
        \end{align*}
        This helps with notation:
        \begin{align*}
            w_j^{n+1} = w_j^n - \frac{\Dt}{\Dx}\Lambda^+\qty(w_j^n - w_{j-1}^n) - \frac{\Dt}{\Dx}\Lambda^-\qty(w_{j+1}^n - w_j^n)
        \end{align*}
        Back to original variables:
        \begin{align*}
            u_j^{n+1} = u_j^n - \frac{\Dt}{\Dx}A^+\qty(u_j^n - u_{j-1}^n) - \frac{\Dt}{\Dx}A^-\qty(u_{j+1}^n - u_j^n)
        \end{align*}
        where $A^+ = V\Lambda^+V^{-1}$ and $A^- = V\Lambda^-V^{-1}$.  \emph{Note: for nonlinear or variable coefficient problems, we would decompose these differences locally ``into left-moving stuff and right-moving stuff''.}

    \section{Conservation Laws and Finite Volume Methods}

        This is not in the book.  The general form of a conservation law is
        \begin{align*}
            u_t + \qty(f(u))_x = 0
        \end{align*}
        where $f$ is a flux function.  If $f(u) = au$, we recover the advection equation.  If $f(u) = \frac{1}{2}u^2$, then $f'(u) = u$, so we recover $u_t + u u_x = 0$ (inviscid Burgers equation (model problem for nonlinear conservation laws - develops shocks etc.~in finite time)). \\

        Remember the integral form of the conservation law is
        \begin{align*}
            \frac{\dd}{\dd t}\int_{x_1}^{x_2}u \dd x = f(u(x_1,t)) - f(u(x_2,t))
        \end{align*}
        If $u$ is a density, the the integral is the ``total amount of stuff'' in the interval $[x_1,x_2]$.  ``Technically, stuff changes only from movement across the boundary.''  How do we deal with an integral conservation law numerically?

        \subsection{Finite Volume Methods}

            In a finite-difference method, when we write $u_j$, we mean this is an approximation of $u(x_j)$.  We are using approximate values of the function in our scheme.  In a finite-volume method, we don't divide the space into a set of points.  Rather, we divide the domain into a set of volumes $c_j$.  In 1D, for finite-difference methods, we consider points $x_j$, but for finite-volume methods, we consider volumes $c_j = \qty[x_{j-\half}, x_{j+\half}]$.  We will let $$u_j \approx \frac{1}{\abs{c_j}}\int_{c_j}u(x)\dd x,$$ i.e.~the average value of the function over the volume $c_j$.  In 1D this looks like
            \begin{align*}
                u_j = \frac{1}{\Dx}\int_{x_{j-\half}}^{x_{j+\half}}u(x)\dd x
            \end{align*}
            Note that
            \begin{align*}
                \frac{1}{\Dx}\int_{x_{j-\half}}^{x_{j+\half}}u(x)\dd x = u(x_j) + \order{\Dx}^2
            \end{align*}
            using midpoint rule and numerical quadrature.  $u$ at the center is a 2nd-order accurate approximation to the average (and vice-versa).

            What is the conservation Law on $c_j$?
            \begin{align*}
                \frac{\dd}{\dd t}\int_{x_{j-\half}}^{x_{j+\half}}u(x)\dd x = f(u({x_{j-\half}}),t) + f(u({x_{j+\half}}),t)
            \end{align*}
            Integrating in time from $t_n$ to $t_{n+1}$ gives
            \begin{align*}
                \Dx\qty(u_j^{n+1} - u_j^n) &= \int_{t_n}^{t_{n+1}} f(u(x_{j-\half},t)) - f(u(x_{j+\half},t)) \dd t
            \end{align*}
            Let $F_{j+\half} \coloneqq \frac{1}{\Dt^2}\int_{t_n}^{t_{n+1}}F(u(x_{j+\half},t))\dd t$.  Also, note
            \begin{align*}
                \Dx\qty(u_j^{n+1} - u_j^n) &= \Dt\qty(F_{j-\half} - F_{j+\half}) \\
                u_j^{n+1} &= u_j^n - \frac{\Dt}{\Dx}\qty(F_{j-\half}{F_{j+\half}}) \\
                \frac{u_j^{n+1} - u_j^n}{\Dt} + \frac{F_{j+\half} - F_{j-\half}}{\Dx} &= 0
            \end{align*}
            We also have
            \begin{align*}
                \frac{1}{\Dx}\int_{x_{j-\half}}^{x_{j+\half}}u(x,t_n)\dd x = u_j^n.
            \end{align*}
            With these averages, this equation is exact - no approximations.

            Next, the approximation enters in the numerical flux function.  Suppose $f(u) = au$.

            \subsubsection{Upwinding}

                \begin{align*}
                    F_{j+\half}^n = \begin{cases}
                        au_j & \text{ if } a \geq 0\\
                        au_{j+1} & \text{ if } a \leq 0
                    \end{cases}
                \end{align*}
                Supposing $a > 0$, then
                \begin{align*}
                    u_j^{n+1} = u_j^n - \frac{\Dt}{\Dx}\qty(a u_j- au_{j-1}^n)
                \end{align*}

            \subsubsection{Two-Step Lax-Wendroff}

                We predict $$u_{j+\half}^{n+\half} = \frac{1}{2}\qty(u_j^n + u_{j+1}^n) - \frac{\Dt}{2\Dx}\qty(f({u_{j+1}^n}) - f^n(u_j)).$$  Anyway,
                \begin{align*}
                    F_{j+\half}^{n+\half} = f(u_{j+\half}^{n+\half})
                \end{align*}
\end{document}












