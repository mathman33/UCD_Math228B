\documentclass{article}


\usepackage[margin=0.6in]{geometry}
\usepackage{amssymb, amsmath, amsfonts}
\usepackage{tabularx}
\usepackage{arydshln}
\usepackage{mathtools}
\usepackage{changepage}
\usepackage{asymptote}
\usepackage{cancel}
\usepackage{physics}
\usepackage{pgf}
\usepackage{enumerate}
\usepackage{placeins}
\usepackage{nth}
\usepackage{array}
\usepackage{tikz}
\usetikzlibrary{arrows,automata}
\tikzset{
  saveuse path/.code 2 args={
    \pgfkeysalso{#1/.style={insert path={#2}}}%
    \global\expandafter\let\csname pgfk@\pgfkeyscurrentpath/.@cmd\expandafter\endcsname
      % not optimal as it is now global through out the document
                           \csname pgfk@\pgfkeyscurrentpath/.@cmd\endcsname
    \pgfkeysalso{#1}},
  /pgf/math set seed/.code=\pgfmathsetseed{#1}}
\usepackage{nicefrac}
\usepackage{pgfplots}
\usepgfplotslibrary{polar}
\pgfplotsset{holdot/.style={fill=white,only marks,mark=*}}
\pgfplotsset{soldot/.style={only marks,mark=*}}
\newcommand{\enth}{$n$th}
\newcommand{\Rl}{\mathbb{R}}
\newcommand{\Cx}{\mathbb{C}}
\newcommand{\sgn}[1]{\text{sgn}\qty[#1]}
\newcommand{\ran}[1]{\text{ran}\qty[#1]}
\newcommand{\E}{\varepsilon}
\newcommand{\qiq}{\qquad \implies \qquad}
\newcommand{\half}{\nicefrac{1}{2}}
\newcommand{\third}{\nicefrac{1}{3}}
\newcommand{\quarter}{\nicefrac{1}{4}}
\newcommand{\f}[3]{#1\ :\ #2 \rightarrow #3}
\newcommand{\Dx}{\Delta x}
\newcommand{\Dy}{\Delta y}
\newcommand{\Dt}{\Delta t}
\newcommand{\Dphi}{\Delta \phi}
\newcommand{\hot}{\text{h.o.t.}}
\newcommand{\centdiff}{\frac{u_j^{n+1} - u_j^n}{\Dt}}
\newcommand{\dod}{Domain of Depdendence}

\newcommand{\tridsym}[3]{
    \qty(\begin{array}{ccccc}
                    #1 & #2 & & & \\
                    #3 & #1 & #2 & & \\
                    & \ddots & \ddots & \ddots &  \\
                    & & #3 & #1 & #2 \\
                    & & & #3 & #1
                \end{array})
}


\DeclareMathOperator*{\esssup}{\text{ess~sup}}

\title{MAT 228B Notes}
\author{Sam Fleischer}
\date{March 13, 2017}

\begin{document}
    \maketitle

    \section{Avoiding Wiggles}

        \begin{itemize}
            \item Suppose $u_j^0 \geq u_{j+1}^0$ for all $j$.  If $u_j^n \geq u_{j+1}^n$ for all $j$ and $n$, then the scheme is monotone-preserving.
            \item Godunov's Theorem - A linear monotonicity preserving scheme is at most first-order accurate.
            \item Idea: To get higher-order accuracy and not get wiggles, we need to use a non-linear scheme.
            \item For example: $$F_{j+\half} = F_{j +\half}^\text{up} + \qty(F_{j+\half}^\text{LW} - F_{j+\half}^\text{up})\phi(u^n)$$ where $\phi$ is a ``flux limiter.''  We have a nonlinear scheme, even for a linear PDE.
            \item Let $u_t + (f(u))_x = 0$.  Godunov's Method (an REA (reconstruct, evolve, average) method) is
            \begin{enumerate}
                \item Reconstruct a function from cell averages.
                \begin{itemize}
                    \item Given $u_j^n$, get $\tilde{u}(x,t_n)$.
                    \item For example: piecewise-constant reconstruction.
                \end{itemize}
                \item Evolve.
                \begin{itemize}
                    \item Solve the PDE exactly using the reconstruction as initial data.
                \end{itemize}
                \item Average on each cell to update the averages.
                \begin{itemize}
                    \item Set $\displaystyle u_J^{n+1} = \frac{1}{\Dx}\int_{x-\half}^{x+\half}\tilde{u}(x,t_{n+1})\dd x$
                \end{itemize}
            \end{enumerate}
            \item Example: Advection with $a > 0$, $\nu \leq 1$.
            \begin{itemize}
                \item Remember $$u_j^{n+1} = u_j^n - \frac{\Dt}{\Dx}\qty(F_{j+\half}^{n} - F_{j-\half}^n).$$
                \item We only need $u(x_{j+\half},t)$ to compute the $u_J^{n+1}$ using the above equation, and $F_{j+\half} = \frac{1}{\Dt}\int_{t_n}^{t_{n+1}}f(\tilde{u}(x_{j+\half},t))\dd t$.
                \item So $\tilde{u}(x_{j+\half},t) = u_j^n$.
                \item $F_{j+\half}^n = au_j^n$ (this is upwinding)
            \end{itemize}
            \item We can generalize to a nonlinear problem $$u_t + \qty(f(u))_x = 0$$ and
            \begin{align*}
                u(x,0) = \begin{cases}
                    u_\ell & x < 0 \\
                    u_r & x > 0
                \end{cases}
            \end{align*}
            This is a Riemann problem.
        \end{itemize}

    \section{Higher-Order Accuracy}
        \begin{itemize}
            \item We can do a more accuracte reconstruction than piecewise constant.  How about piecewise linear?  But remember we need to preserve the averages.
            \begin{align*}
                \tilde{u}(x,t_n) = u_j^n + \sigma_j^n\underbrace{\qty(x - x_j)}_{0 \text{ average}} \qquad x_{j-\half} \leq x \leq x_{j+\half}
            \end{align*}
            \item So the average on the cell is independent from $\sigma$ (independent of slope).
            \item What slope do we pick?
            \begin{itemize}
                \item $\sigma_j^n = 0$ this is upwinding
                \item $?$ force continuity?
                \item $\sigma_j^n = \frac{u_{j+1}^n - u_{j-1}^n}{2\Dx}$ (centered, Fromm's method)
                \item $\sigma_j^n = \frac{u_{j+1}^n - u_j^n}{\Dx}$, downward slope (Lax-Wendroff)
                \item $\sigma_j^n = \frac{u_j - u_{j-1}}{\Dx}$ (upwind difference, beam-warming).
            \end{itemize}
            \item Considering $a > 0$, using LW slope for
            \begin{align*}
                u_j^n = \begin{cases}
                    1 & j \leq J \\
                    0 & j > J
                \end{cases}
            \end{align*}
            This immediately introduces an overshoot behijnd the jump, killing monotonicity.  So Lax-Wendroff gives wiggles behind, Fromm's method gives wiggles on both siedes (but they're half as off), and Beam-Warming gives wiggles in front of the discontinuity.
            \item BUT!  You can pick different slopes at each interval!
            \item No wiggles if we pick Lax-Wendroff to the right of the jump and beam-warming to the left.  To avoid oscillations, use so-called ``limited slope''
            \item ``Minmod'' slope
            \begin{align*}
                u_j^n = \text{minmod}\qty(\frac{u_j - u_{j-1}}{\Dx},\frac{u_{j-1} - u_j}{\Dx})
            \end{align*}
            where
            \begin{align*}
                \text{minmod}(a,b) = \min(\abs{a},\abs{b})
            \end{align*}
            A different choice monotonized centered slope
            \begin{align*}
                \sigma_j^n = \text{minmod}\qty(\frac{u_{j+1}-u_{j-1}}{2\Dx},2\qty(\frac{u_{j+1} - u_j}{\Dx}),2\qty(\frac{u_j - u_{j-1}}{\Dx}))
            \end{align*}
        \end{itemize}

\end{document}












