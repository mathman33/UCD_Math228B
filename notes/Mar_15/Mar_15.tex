\documentclass{article}


\usepackage[margin=0.6in]{geometry}
\usepackage{amssymb, amsmath, amsfonts}
\usepackage{tabularx}
\usepackage{arydshln}
\usepackage{mathtools}
\usepackage{changepage}
\usepackage{asymptote}
\usepackage{cancel}
\usepackage{physics}
\usepackage{pgf}
\usepackage{enumerate}
\usepackage{placeins}
\usepackage{nth}
\usepackage{array}
\usepackage{tikz}
\usetikzlibrary{arrows,automata}
\tikzset{
  saveuse path/.code 2 args={
    \pgfkeysalso{#1/.style={insert path={#2}}}%
    \global\expandafter\let\csname pgfk@\pgfkeyscurrentpath/.@cmd\expandafter\endcsname
      % not optimal as it is now global through out the document
                           \csname pgfk@\pgfkeyscurrentpath/.@cmd\endcsname
    \pgfkeysalso{#1}},
  /pgf/math set seed/.code=\pgfmathsetseed{#1}}
\usepackage{nicefrac}
\usepackage{pgfplots}
\usepgfplotslibrary{polar}
\pgfplotsset{holdot/.style={fill=white,only marks,mark=*}}
\pgfplotsset{soldot/.style={only marks,mark=*}}
\newcommand{\enth}{$n$th}
\newcommand{\Rl}{\mathbb{R}}
\newcommand{\Cx}{\mathbb{C}}
\newcommand{\sgn}[1]{\text{sgn}\qty[#1]}
\newcommand{\ran}[1]{\text{ran}\qty[#1]}
\newcommand{\E}{\varepsilon}
\newcommand{\qiq}{\qquad \implies \qquad}
\newcommand{\half}{\nicefrac{1}{2}}
\newcommand{\third}{\nicefrac{1}{3}}
\newcommand{\quarter}{\nicefrac{1}{4}}
\newcommand{\f}[3]{#1\ :\ #2 \rightarrow #3}
\newcommand{\Dx}{\Delta x}
\newcommand{\Dy}{\Delta y}
\newcommand{\Dt}{\Delta t}
\newcommand{\Dphi}{\Delta \phi}
\newcommand{\hot}{\text{h.o.t.}}
\newcommand{\centdiff}{\frac{u_j^{n+1} - u_j^n}{\Dt}}
\newcommand{\dod}{Domain of Depdendence}

\newcommand{\tridsym}[3]{
    \qty(\begin{array}{ccccc}
                    #1 & #2 & & & \\
                    #3 & #1 & #2 & & \\
                    & \ddots & \ddots & \ddots &  \\
                    & & #3 & #1 & #2 \\
                    & & & #3 & #1
                \end{array})
}


\DeclareMathOperator*{\esssup}{\text{ess~sup}}

\title{MAT 228B Notes}
\author{Sam Fleischer}
\date{March 15, 2017}

\begin{document}
    \maketitle

    \section{Flux Limiters}

        \begin{itemize}
            \item Set $\tilde{u} = u_j^n + \sigma_j^n\qty(x - x_j)$.
            \item We have a lot of freedom over how we choose the slope on each cell.
            \item Consider the advection equation with $a > 0$.
            \item We compute the flux through the $j - \half$ edge.
            \begin{align*}
                F_{j-\half}^n &= \frac{1}{\Dt}\int_{t_n}^{t_{n+1}}f(\tilde{u}(x_{j-\half},t))\dd t \\
                &= \frac{1}{\Dt}\int_{t_n}^{t_{n+1}}a\tilde{u}(x_{j-\half},t)\dd t \\
                &= \frac{1}{\Dt}\int_{t_n}^{t_{n+1}}a\qty(u_{j-1}^n + \sigma_{j-1}^n\qty(x_{j-\half} - a\qty(t - t_n) - x_{j-1}))\dd t \\
                &= au_{j-1}^n + a\sigma_{j-1}^n\qty(\frac{\Dx}{2} - a\frac{\Dt}{2}) \\
                &= \underbrace{au_{j-1}^n}_\text{upwind flux} + \underbrace{\frac{a}{2}\qty(1 - \nu)\Dx\sigma_{j-1}^n}_\text{second-order correction}
            \end{align*}
            where $\nu = \frac{a\Dt}{\Dx}$ is the Courant number.
            \item For Lax-Wendroff,
            \begin{align*}
                \Dx\sigma_{j-1}^n = u_j - u_{j-1} = \qty(\Delta u)_{j-\half}
            \end{align*}
            \item For $a$ positive or negative, we can write
            \begin{align*}
                F_{j-\half}^n = F_{j-\half}^\text{up} + \frac{\abs{a}}{2}\qty(1 - \abs{\nu})\delta_{j-\half}^n
            \end{align*}
            where $\delta_{j-\half}^n$ is a limited difference that depends on the solution.
            \item We need a way to measure smoothness of the solution in order to decide $\delta$.
            \item Introduce $\theta_{j-\half}$ be a ratio of successive differences
            \begin{align*}
                \theta_{j-\half} = \frac{\Delta u_{J_\text{up}-\half}}{\Delta u_{j-\half}}
            \end{align*}
            where
            \begin{align*}
                J_\text{up} = \begin{cases}
                    j-1 & a > 0 \\
                    j+1 & a < 0
                \end{cases}
            \end{align*}
            \item For smooth functions away from extreme points, we expect $\theta\approx1$.
            \item Now we let
            \begin{align*}
                \delta_{j-\half}^n = \phi\qty(\theta_{j-\half})\qty(\Delta u)_{j-\half}
            \end{align*}
            and $\phi$ is called the flux limiter function.
            \item What are good choices of $\phi$?
            \begin{itemize}
                \item If $\phi = 0$ we get upwinding.
                \item If $\phi = 1$ we get Lax-Wendroff.
                \item If $\phi(\theta) = \theta$ we get Beam Warming
                \item ``High resolution schemes'' minmod: $\phi(\theta) = \text{minmod}(1,\theta)$.  This picks between the above three.  The problem is that this is really diffusive.
                \item We have the ``monotonized-centered'' (MC) flux limiter $\phi(\theta) = \max(0,\min(\frac{1+\theta}{2},2,2\theta))$.  This is less diffusive.
                \item Superbee: $\phi(\theta) = \max(0,\min(1,2\theta),\min(2,\theta))$.  This is the most sharpening.
                \item Van Leer: $\phi(\theta) = \dfrac{\theta + \abs{\theta}}{1 + \abs{\theta}}$.
            \end{itemize}
        \end{itemize}

    \section{Simulations}

        \begin{itemize}
            \item Upwinding had diffusion but small phase error
            \item Lax Wendroff has bad (negative) phase error for hogh frequencies with little diffusion.
            \item Beam Warming has bad (positive) phase error for high frequencies with little diffusion.
            \item Minmod has much less diffusion than upwinding, very little phase error.  The most diffusive of the ``high resolution schemes.''
            \item MC limiter has even less diffusion than minmod.  Also very little phase error.
            \item Superbee looks amazing.  It preserves sharpness the best.  The problem is that superbee sharpens smooth things that shouldn't be sharp.  We saw that after testing for 10 periods.
            \item Van Leer is the most general ``middle of the road'' limiter function.
        \end{itemize}
        All of the high resolution methods are designed to be second-order on smooth data and to avoid introducing unphysical oscillations. \\

    \section{Total Variation}

        The total variation of a grid function
        \begin{align*}
            \text{TV}(\underline{u}) = \sum_j\abs{u_{j+1} - u_j}
        \end{align*}
        The total variation of a differentiable function $f$ is
        \begin{align*}
            \text{TV}(f) = \int_a^b\abs{f'(x)}\dd x
        \end{align*}
        So $f(x) = 0$ is no variation.  $f(x) = \sin(0.1x)$ is small variation.  $f(x) = \sin(10x)$ is high variation.

        Let $f_k(x) = e^{ikx}$ on $[0,2\pi)$.  Then
        \begin{align*}
            \text{TV}(f_k) = \int_0^{2\pi}\abs{ike^{ikx}}\dd x = 2\pi\abs{k}
        \end{align*}
        Next class we will derive the hig-res schemes as natural ways to reduce total variation in time.

\end{document}












